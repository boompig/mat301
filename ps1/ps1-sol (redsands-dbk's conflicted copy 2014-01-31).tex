\documentclass[a4paper,12pt]{article}

\usepackage{amsmath}
\usepackage{amsfonts}
\usepackage{amssymb}
\usepackage{amsthm}

\usepackage{cancel} % to do the blank symbol
\usepackage{enumitem} % allows indexing lists from 0

\usepackage{listings} % for code
\usepackage{color}

\lstset{tabsize=4}

\newtheorem*{proofSketch}{Proof Sketch}
\newtheorem{lemma}{Lemma}

\usepackage{mathtools} % for paired delimiters
\DeclarePairedDelimiter{\set}{\lbrace}{\rbrace}
\DeclarePairedDelimiter{\gen}{\langle}{\rangle}
\DeclarePairedDelimiter{\ceil}{\lceil}{\rceil}
\DeclarePairedDelimiter{\floor}{\lfloor}{\rfloor}

\renewcommand{\thesubsection}{\thesection (\alph{subsection})} % subsections should have letters

\lstset {
numbers=left,
stepnumber=1,
tabsize=2,
language=Python,
breaklines=true
%basicstyle=\footnotesize
}

\begin{document}

\begin{titlepage}
\title{MAT301 Problem Set 1}
\author{Daniel Kats \\ 997492468}
\clearpage
\maketitle
\thispagestyle{empty}
\end{titlepage}

\section{Question 1}
Let $G = \{ ( t, x ) \mid| t, x \in \mathbb{R}, t \neq 0 \}$. For $( t_1 , x_1 )$ and $( t_2 , x_2 )$ in $G$, define:

\begin{equation*}
( t_1 , x_1 ) \ast ( t_2 , x_2 ) = ( t_1 t_2 , t_1 x_2 + x_1 / t_2 )
\end{equation*}

\subsection{}

Prove that $G$ is a group with respect to the operation *.

\begin{proof}
I will cover each group axiom in order:

\subsubsection*{Closure}

Suppose $a = (t_1, x_1), b = (t_2, x_2) \in G$. 

\begin{equation*}
a \ast b = ( t_1 , x_1 ) \ast ( t_2 , x_2 ) = (t_1 t_2 , t_1 x_2 + x_1 / t_2 )
\end{equation*}

Since $t_1 \neq 0$ and $t_2 \neq 0$ by definition, both the first and second elements of the resultant tuple are well defined and real. Moreover, $t_1t_2 \neq 0$. Thus $a \ast b \in G$.

\subsubsection*{Identity and Non-emptiness}
Let $e = (1, 0)$. Clearly $e \in G$, so $G$ is non-empty. Consider any element $a = (t, x) \in G$. Then $a \ast e = (t, t \cdot 0 + x / 1) = (t, x) = a$. Thus $G$ has an identity element, and it is $e$.

\subsubsection*{Inverse}
I claim that if $a = (t, x)$, then $b = (\frac{1}{t}, -x)$ is an inverse for $a$. $a \ast b = (t \cdot \frac{1}{t}, -tx + tx) = (1, 0) = e$. Notice that as long as $t \neq 0$, the inverse exists, which is consistent with our definition of $G$.

\subsubsection*{Associativity}
Let $a = (t_1, x_1), b = (t_2, x_2), c=(t_3, x_3) \in G$.

\begin{equation*} 
(a \ast b) \ast c = ((t_1 t_2, t_1 x_2 + x_1 / t_2)) \ast (t_3, x_3)
\end{equation*}
\begin{equation*}
= (t_1 t_2 t_3, t_1 t_2 x_3 + \frac{t_1 x_2 + x_1 / t_2}{t_3})
\end{equation*}
\begin{equation} \label{1a:1}
=  (t_1 t_2 t_3, t_1 t_2 x_3 + \frac{t_1x_2}{t_3} + \frac{x_1}{t_2t_3})
\end{equation}

On the other hand:

\begin{equation*}
a \ast (b \ast c) = (t_1, t_2) \ast ((t_2 t_3, t_2 x_3 + x_2 / t_3))
\end{equation*}
\begin{equation*}
= (t_1 t_2 t_3, t_1 \cdot (t_2 x_3 + x_2 / t_3) + \frac{x_1}{t_2t_3})
\end{equation*}
\begin{equation} \label{1a:2}
= (t_1 t_2 t_3, t_1t_2x_3 + \frac{t_1x_2}{t_3} + \frac{x_1}{t_2t_3})
\end{equation}

And notice that \eqref{1a:1} and \eqref{1a:2} are the same.

\end{proof}

\subsection{}

Find all elements belonging to the centre $Z(G)$ of $G$. 

The elements in the center are all those elements $z = (a, b), a \neq 0$ which satisfy the equation:

\begin{equation}
tx = a^2tx + ab - t^2ab
\end{equation}

For all $tx$ and $t \neq 0$. Note that if $a \neq \pm 1, a \neq 0$, then our equation for $z$ depends on $x$, which cannot be the case. So $a = \pm 1$. Then we have the equation $ab = t^2ab$, which must be true for all non-zero values of $t$. This can only be accomplished by setting $b$ to 0.

Thus the center is $Z(G) = \set{(1, 0), (-1, 0)}$.

\section{Question 2}

Let $G = D_6$ (the dihedral group of order 12). Let $r$ be a fixed rotation in $G$ such that $\left|r\right| = 6$ and let $s$ be a fixed reflection in $G$.

\subsection{}

Let $H$ be the smallest subgroup of $G$ that contains $rs$ and $sr^3$ . List all of the elements in $H$.

The elements of $H$ are: $\{ e = r^0, r^2, r^4, rs, sr, sr^3 \}$. Notice that we have the following identities:

\begin{equation*}
sr^3 = r^3s, (rs)^2 = e, (sr^3)^2 = e, r^2r^4 = e, (r^5sr^3s), (sr)^2 = e
\end{equation*}

We can see that $H$ has closure, the identity element, and inverses. Also $H$ is non-empty. Each of these elements is essential for closure or identity properties, so no smaller $H$ can be found.

\subsection{}

Find an Abelian subgroup $H^{\prime}$ of $D_6$ that contains exactly 2 reflections.

Let $r$ be a rotation of order 2. Let $s_1$ and $s_2$ be two reflections: $s_1$ is along the vertical axis, and $s_2$ is along the horizontal axis. Then let $H^{\prime} = \{e, r, s_1, s_2\}$.

Then we have $rs_1 = s_2$, and $rs_2 = s_1$. This group is also Abelian, and each element is self-inverse.

I will give a quick outline of why this group is Abelian. First $s_1rs_1 = s_1s_1r^{-1} = er = r$. But $rs_1s_1 = re = r$. Now $s_1s_2 = s_1rs_1$ as before. We get similar results for $s_2$.

\section{Question 3}

\subsection{}

Let $G$ be the group of functions from $\mathbb{Z}_{15}$ to $\mathbb{Z}_{15}$, under the operation $(f_1 \star f_2)(m) = (f_1(m) + f_2(m)) \pmod{15}, m \in \mathbb{Z}_{15}$. Let $H = \{f \in G \mid f(m) \text{ is even for all } m \in \mathbb{Z}_{15}\}$.

I do not think $H$ is a subgroup, because I do not think inverses are well-defined, as $f \in H$ is by definition not onto. So by definition, there will be some elements $m \in \mathbb{Z}_15$ and $n \in \mathbb{Z}_15, m \neq n$ such that $f(m) = f(n)$. Thus $f^{-1}(f(n)))$ is not well-defined.

\subsection{}

Let $G=GL(2, \mathbb{R})$ and let $H=\left\{A=\left[ \begin{array}{ccc} a+b & -2b \\ b & a-b \\ \end{array} \right] \in G \mid a^2 + b^2 =1 \right\}$.

First, $H$ is non-empty, as setting $a = 0$ and $b = 1$ creates a valid matrix in $H$. Next, note that:

\begin{equation*}
det(A) = (a^2 - b^2) + 2b^2 = a^2 + b^2 = 1
\end{equation*}

For the inverse, if $B \in H$, then we get:

\begin{equation*}
B^{-1} = \frac{1}{det(B)} \left[\begin{array}{cc}
a - b 	&		-(-2b) \\
 -(b)			&		a+b
\end{array}
\right] = \left[ \begin{array}{cc}
a - b 	&		2b \\
 -b			&		a+b
\end{array} \right]
\end{equation*}

Now suppose that $A \in H$ and $B \in H$. We know that multiplying the matrices together will result in some valid matrix in $GL(2, \mathbb{R})$, by properties of matrices. All we have to check is that $det(AB^{-1}) = det(A)det(B^{-1}) = \frac{det(A)}{det(B)} = 1$. Which is exactly the property we need. Thus $H$ is a subgroup of $G$.

\subsection{}

Let $G$ be the group of nonzero real numbers under multiplication and let $H = \{a+b\sqrt{2} \mid| a,b\in \mathbb{Z}, \text{at least one of $a$ and $b$ is nonzero}\}$.

Let $a = 2$ and $b = 2$. Then $c = a + b\sqrt{2} \in H$, and the value of $c$ is $2 + 2\sqrt{2}$. Notice that the inverse of $2 + \sqrt{2}$ is $1 + \frac{-1}{2} \sqrt{2}$. There is no way to express the second number as an integer. Therefore $c^{-1} \notin H$. Thus $H$ cannot be a subgroup of $G$.

\section{Question 4}
Let $S$ be a subset of a group $G$. If $a \in G$, let $aSa^{-1} = \{ asa^{-1} \mid s \in S \}$.

\subsection{}
Prove that $S$ is a subgroup of $G$ if and only if $aSa^{-1}$ is a subgroup of $G$

\begin{proof}
$\Leftarrow$: Suppose that $S$ is a subgroup of $G$. Also suppose $a \in G$. Since $G$ is a group, $a^{-1} \in G$ also. Now take any element $s \in S$. Since $S$ is a subgroup of $G$, $s \in G$. Therefore $as \in G$ by properties of groups. And also $asa^{-1} \in G$. This is true for any generic element $s \in S$. Therefore $aSa^{-1}$ is a subgroup of $G$.

$\Rightarrow$: Suppose that $aSa^{-1}$ is a subgroup of $G$, for some $a \in G$. Take arbitrary $b \in aSa^{-1}$. By definition, $b = asa^{-1}$ for some element $s$. I will show that $s \in G$ also. $ba \in G$ by properties of groups, and $ba = as$. Similarly $a^{-1} \in G$ as before, and $a^{-1}ba \in G$. So $s \in G$.
\end{proof}

\subsection{}

Suppose that $G = D_n$ $(n \geq 3)$ and $S$ is the set of all reflections in $G$. Prove that
$aSa^{-1} = S$ for all $a \in G$.

\begin{proof}
There are two cases.

Case 1: $a$ is a rotation
Suppose that $a$ is some rotation, and $s$ is some reflection. Then $asa^{-1} = a^2s$, after left-multiplying both sides by $a$. Notice that $a^2$ is actually a rotation. But by another property of dihedral groups:

\begin{equation*}
\text{any rotation} \cdot \text{any reflection} = \text{some reflection}
\end{equation*}

This property follows intuitively if we notice that there are only two types of elements in $D_n$: rotations and reflections. And if we color the "top" of the polygon white and the bottom black, then rotations maintain the color, while reflections flip the color. And since $D_n$ has closure, any compound operation must be either a rotation or reflection. Thus for any rotation $r$, $rs$ must be a reflection, since it changes the color of the polygon. In particular, $a^2s$ is a reflection, so it is in $S$.

Case 2: $a$ is a reflection
Suppose that $a$ is some reflection, and $s$ is some reflection. Then $asa^{-1} = a^2s$, after left-multiplying both sides by $a$. However $a^2 = e$, by properties of reflections. Therefore $a^2s = s$, so trivially $asa^{-1} \in S$. 
\end{proof}

\section{Question 5}
Let $U (16) = \{ 1, 3, 5, 7, 9, 11, 13, 15 \}$. This is a group under the binary operation
of multiplication modulo 16. That is, $m ∗ n = mn \pmod{16}$.

\subsection{}
Find all elements in the cyclic subgroup $\gen{3}$. 

We have $\{ 11, 9, 3, 1 \}$.

\subsection{}

Find an element $m \in U (16)$ such that $|m| = 4$ and $| \gen{m} \cap \gen{3} | = 2$. Is $m$ unique?

Both 5 and 13 have this property. They are both in $U(16)$ (this is given). The order of 5 is 4, since $5^4 = 625 = 1 \pmod{16}$. Similarly $13^4 = 28561 = 1 \pmod{16}$. Finally:

\begin{equation}
\gen{5} = \gen{13} = \set{1, 5, 9, 13}
\end{equation}

And the intersection of these sets with $\gen{3}$ is clearly just $\set{1, 9}$, which has order 2. In conclusion, $m$ exists and is not unique.

\subsection{}

Determine whether $U (16)$ is a cyclic group. 

I explicitly computed the order of each element, and none had order 8. Thus $U(16)$ is not cyclic.

\section{Question 6}
Let $a$ and $b$ be elements of a group $G$. Assume that both $a$ and $b$ have finite order.

\subsection{}

Prove that if $ab = ba$ and $gcd(|a|, |b|) = 1$, then $|ab| = |a||b|$.

\begin{proof}
It is clear that if $ab = ba$, then for any integer $n > 0$, $(ab)^n = a^nb^n$. By induction on $n$: if $n = 1$, then this is trivial. If $n > 1$, then:

\begin{equation}
(ab)^n = (ab)^{n-1}ab = a^{n - 1}b^{n - 1} = a^{n - 1}b^{n-1}ba = a^{n - 1}b^na
\end{equation}

But trivially $b^n$ is also an element of $G$, so we can apply the rule again.

\begin{equation}
a^{n - 1}b^na = a^{n-1}ab^n = a^nb^n
\end{equation}

Let $|a| = p$ and $|b| = q$. Let $n = pq$. This means $(ab)^n = (a^p)^q(b^q)^p = e^qe^p = e$. We know by theorem 4.1 Corollary 2 that $|ab|$ divides $n$. However, $n = pq$ and $gcd(p, q) = 1$ by the problem statement. Therefore $|ab|$ must be $n$.
\end{proof}

\subsection{}

Find an example of elements $a$ and $b$ in a particular group $G$ such that $a \neq e$,
$b \neq e$, $gcd(|a|, |b|) = 1$ and $|ab| = |a|$.

Consider $D_3$. Let $r$ be a rotation such that $|r| = 3$. Let $s$ be some reflection such that $|s| = 2$. $gcd(2, 3) = 1$. Then $sr$ is a reflection (by properties of Dihedral groups), and thus has order 2, the same order as $s$. So let $a = s$ and $b = r$. We then get $|ab| = |sr| = 2$ and $|a| = |s| = 2$.

\end{document}
