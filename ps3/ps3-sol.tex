\documentclass[a4paper,12pt]{article}
 
 %%%%%%%%%%%%%%%%%%%%%%%%%%%%%% STANDARD PREAMBLE %%%%%%%%%%%%%%%%%%%%%%%%%%%
\usepackage{amsmath}
\usepackage{amsfonts}
\usepackage{amssymb}
\usepackage{amsthm}

%\usepackage{cancel} % to do the blank symbol
%\usepackage{enumitem} % allows indexing lists from 0

\usepackage{listings} % for code
\usepackage{color}

\lstset{tabsize=4}

\newtheorem*{proofSketch}{Proof Sketch}
\newtheorem{lemma}{Lemma}

\usepackage{mathtools} % for paired delimiters
\DeclarePairedDelimiter{\set}{\lbrace}{\rbrace}
\DeclarePairedDelimiter{\gen}{\langle}{\rangle}
\DeclarePairedDelimiter{\ceil}{\lceil}{\rceil}
\DeclarePairedDelimiter{\floor}{\lfloor}{\rfloor}
\DeclarePairedDelimiter{\norm}{\lvert}{\rvert}

\renewcommand{\thesubsection}{\thesection.(\alph{subsection})} % subsections should have letters

\lstset {
numbers=left,
stepnumber=1,
tabsize=2,
language=Python,
breaklines=true
%basicstyle=\footnotesize
}

\numberwithin{equation}{section}

 %%%%%%%%%%%%%%%%%%%%%%%%%%%%%% END PREAMBLE %%%%%%%%%%%%%%%%%%%%%%%%%%%

\begin{document}

\begin{titlepage}
\title{MAT301 Problem Set 3}
\author{Daniel Kats \\ 997492468}
\clearpage
\maketitle
\thispagestyle{empty}
\end{titlepage}

\section{Question 1}

\subsection{}

Suppose that $x = \left( \begin{array}{cc}
a & c \\
0 & 1 \end{array} \right) \in G$. Then we have

\begin{equation}\label{1a:xH}
xH = \left( \begin{array}{cc}
a & c \\
0 & 1 \end{array} \right)\left( \begin{array}{cc}
1 & b \\
0 & 1 \end{array} \right) = 
\left( \begin{array}{cc}
a & ab + c\\
0 & 1
\end{array} \right)
\end{equation}

Also

\begin{equation}\label{1a:Hx}
Hx = \left( \begin{array}{cc}
1 & b \\
0 & 1 \end{array} \right)\left( \begin{array}{cc}
a & c \\
0 & 1 \end{array} \right) = 
\left( \begin{array}{cc}
a & b + c\\
0 & 1
\end{array} \right)
\end{equation}

Note that \eqref{1a:Hx} is not equal to \eqref{1a:xH} when $a \neq 1$ and $b \neq 0$. Therefore $H$ is not a normal subgroup of $G$.

\subsection{}
Consider $x = (1, 2, 3) \in S_4$. Then $x^{-1} = (3, 2, 1)$. Using the normal subgroup test from Theorem 9.1 in the book, we look at $xHx^{-1}$ and see if this is a subset of $H$. We know that any element in $H$ has the form $(1, 3, 4)^n$ where $n$ is a natural number. Consider the case where $n = 2$

\begin{equation}
(1, 2, 3)(1, 3, 4)^2(3, 2, 1) = (1, 2, 3)(1, 4, 3)(3, 2, 1) = (1, 2, 4) \notin H
\end{equation}

Therefore $xHx^{-1} \not\subset H$, so $H$ is not a normal subgroup of $G$.

\section{Question 2}

%(a)
\subsection{}
By Lagrange's Theorem, the number of distinct cosets is precisely $|G| / |H|$. Moreover, we can see that $H = \gen{69} = \gen{3}$. Therefore the order of the factor group $G / H$ is simply 3.

%(b)
\subsection{}
The element $46 + \gen{69}$ is the same as the element $46 + \gen{3}$, and is actually the same as the element $1 + \gen{3}$. Therefore, the order of this element is 3.

\section{Question 3}

\begin{proof}
Suppose that $H$ is a normal subgroup of $S_4$ such that $|H| = 8$. I will show that every element of order 2 in $S_4$ must belong to $H$.

Since $H$ is normal, the factor group $S_4 / H$ exists and has order 3 (by Lagrange's Theorem, and because $|S_n| = n!$). Since the order of each element divides the order of the group, for all $\alpha \in S_4$, it is the case that $(\alpha H)^3 = \alpha^3 H = H$ (because each coset of the form $\alpha H$ is actually an element of the factor group). This means that $\alpha^3 \in H$ for all $\alpha \in S_4$.

Any $\alpha$ of order 2 has the property that $\alpha^3 = \alpha$. So $H$ contains all elements of order 2. But there are at least 9 of these: 6 of the form $(a, b)$ and at least 3 of the form $(a, b)(c, d)$. Contradiction, because cardinality of $H$ should be 8 < 9.

\end{proof}

\section{Question 4}

% (a)
\subsection{}

I will show $H^{\prime}$ is a subgroup of $G$.

\begin{proof}
Consider $e \in G$. Clearly $e \in H$, and $e^2 = e$, so $e \in H^{\prime}$, so $H^{\prime}$ is non-empty. Not consider any two elements $a, b \in H^{\prime}$. Clearly $a, b \in G$ and we have $a^2, b^2 \in H$ by definition. Notice that because $G$ is a group, $ab^{-1} \in G$. All that remains to be proven is that $ab^{-1}ab^{-1} \in H$. 

Now, suppose that $G / H$ is Abelian. Then for every $a, b \in G$, we have:

\begin{equation}
(aH)(bH) = (bH)(aH)
\end{equation}

\begin{equation}
abH = baH
\end{equation}

\begin{equation}
abH = baH
\end{equation}

This means that for every $x \in H$, there exists $y \in H$ such that

\begin{equation}
abx = bay
\end{equation}

\begin{equation}
(ba)^{-1}abx = y
\end{equation}

Therefore it must be the case that $(ba)^{-1}abx \in H$. Since $x \in H$ and $H$ is a group, $(ba)^{-1}ab \in H$. And by closure of $H$, we can multiply this quantity on the left by $a^2$ and on the right by $b^{-2}$, and we get $ab^{-1}ab^{-1} \in H$. So we have shown that $H^{\prime}$ is a subgroup of $G$.

\end{proof}

I will show $H^{\prime}$ is normal in $G$.
\begin{proof}
I will show that for every $g \in G$ and $a \in H^{\prime}$, it follows that $gag^{-1} \in H^{\prime}$. First, it is trivial that $gag^{-1} \in G$, since $G$ is a group. Now I must show $(gag^{-1})^2 \in H$. $(gag^{-1})^2 = ga^2g^{-1}$. But this is trivial when $a^2 \in H$, since we know $H$ is normal, which is precisely this statement.
\end{proof}

\section{Question 5}

% (a)
\subsection{}

Suppose $x = a^i \in G$ and $y = a^j \in G$ where $i$ and $j$ are non-negative integers.

\begin{equation}
\phi (x) \phi(y) = \phi(a^i)\phi(a^j) = b^{15i}b^{15j} = b^{15(i + j)}
\end{equation}

\begin{equation}
\phi(xy) = \phi(a^ia^j) = \phi(a^{i + j}) = b^{15(i + j)}
\end{equation}

Therefore $\phi$ is a homomorphism by definition.

Since the order of $G^{\prime}$ is 18, we have $b^{18k} = e$ where $k$ is a non-negative integer. So we are looking for all non-negative integers $j$ such that $15j = 18k$. $j = \frac{6k}{5}$. Since $j$ must be an integer, $k$ must be divisible by 5, and so $j$ may be any multiple of 6. However, note that the order of $G$ is 12, so there are actually only 2 elements in $G$ that fit this criteria: $a^0 = e$ and $a ^ 6$.

\begin{equation}
Ker(\phi) = \{e, a^6\}
\end{equation}

%(b)
\subsection{}

Let $x, y \in \mathbb{R}$.

\begin{equation*}
\phi(x)\phi(y) = 
\left( 
	\begin{array}{cc}
	cos(x) & -sin(x) \\
	sin(x) & cos(x) 
	\end{array} 
\right)
\left( 
	\begin{array}{cc}
	cos(y) & -sin(y) \\
	sin(y) & cos(y) 
	\end{array} 
\right)
\end{equation*}

\begin{equation*}
=\left( 
	\begin{array}{cc}
	cos(x)cos(y) - sin(x)sin(y) & -cos(x)sin(y) - sin(x)cos(y) \\
	sin(x)cos(y) + cos(x)sin(y)  & -sin(x)sin(y) + cos(x)cos(y)
	\end{array} 
\right)
\end{equation*}
\begin{equation}
= 
\left( 
	\begin{array}{cc}
	cos(x + y) & -sin(x + y) \\
	sin(x + y)  & cos(x + y)
	\end{array} 
\right)
\end{equation}

\begin{equation}
\phi(x + y) = \left( 
	\begin{array}{cc}
	cos(x + y) & -sin(x + y) \\
	sin(x + y)  & cos(x + y)
	\end{array} 
\right)
\end{equation}

Therefore $\phi$ is a homomorphism by definition. Note that this is under the assumption that the group operation is addition in the reals.

The identity in $G^{\prime}$ is $\left( 
	\begin{array}{cc}
	1 & 0 \\
	0  & 1
	\end{array} 
\right)$ so $cos(x) = 1$ and $sin(x) = 0$. Therefore the kernel consists of all $x$ that are an integer multiple of $2\pi$.

\begin{equation}
Ker(\phi) = \{ 2\pi k \mid k \in \mathbb{Z} \}
\end{equation}

\section{Question 6}

%(a)
\subsection{}

Prove that $\phi(s)$ is a reflection.

\begin{proof}

Consider the property of dihedral groups where $r$ is any rotation and $s$ is any reflection: $rs = sr^{-1}$. In particular, note that $\phi(rs) = \phi(sr^{-1})$

\begin{equation}
\phi(rs) = \phi(r)\phi(s) = r^{12}\phi(s)
\end{equation}

\begin{equation}
\phi(sr^{-1}) = \phi(s)[\phi(r)]^{-1} = \phi(s)(r^{12})^{-1} = \phi(s)r^8
\end{equation}

Note that these two quantities must equal each other.

\begin{equation}
r^{12}\phi(s) = \phi(s)r^{8}
\end{equation}

Note that $\phi(s) \neq e$, since $r^{12} \neq r^8$ (which we know because $|r| = 20$). 

Suppose that $\phi(s)$ is a rotation. Note that rotations commute, so we have $\phi(s)r^{12} = \phi(s)r^{8}$. Now we are able to left-multiply by $\phi(s)^{-1}$ (which must exist because $D_{20}$ is a group) to obtain $r^{12} = r^8$, which is a contradiction. Therefore $\phi(s)$ cannot be a rotation either. It follows that $\phi(s)$ must be a reflection.

\end{proof}

%(b)
\subsection{}

Because of part (a), we know that no reflection is in the kernel. For any rotation $x$, note that $x = r^k$ where $k$ is a non-negative integer. We have 

\begin{equation}
\phi(x) = \phi(r^k) = [\phi(r)]^k = r^{12k}
\end{equation}

If we have $r^{12k} = e$, then $12k$ must be divisible by 20. It follows that $3k$ must be divisible by 5, so $k$ must be a multiple of 5.

\begin{equation}
Ker(\phi) = \{e, r^5, r^{10}, r^{15} \}
\end{equation}

These are the only elements in the kernel, because $D_{20}$ contains only reflections and rotations (and identity maps to identity under all homomorphisms).

%(c)
\subsection{}

\begin{proof}
Let $s$ be any reflection from $D_{20}$, and let $r$ be the rotation of degree 20 from $D_{20}$. 

On the one hand, we have:

\begin{equation}
sr\{e, r^5, r^{10}, r^{10} \} = \{ sr, sr^6, sr^{11}, sr^{16} \}
\end{equation}

On the other hand, we have:

\begin{equation}
rs \{e, r^5, r^{10}, r^{10} \} = sr^{-1} \{e, r^5, r^{10}, r^{10} \} =  \{ sr^{19}, sr^4, sr^{9}, sr^{14} \}
\end{equation}

Clearly these two sets are not equal, so the factor group is not Abelian.
\end{proof}

%(d)
\subsection{}
Consider the Theorem 10.3 from the textbook. Since $\phi$ is a homomorphism from $D_{20}$ to $D_{20}$, we have that there exists an isomorphism between $D_{20} / Ker \phi$ and $D_{20}$ given by $gKer\phi \rightarrow \phi(g)$. This works because $\phi$ is onto. $m$ in this case is 20.

\section{Question 7}

%(a)
\subsection{}
 
I will show that $G$ is not Abelian.

\begin{proof}
Suppose that $G$ is Abelian. Consider any two elements $x, y \in S_4$. Because $\phi$ is onto, there exist elements $a, b \in G$ such that $\phi(a) = x$ and $\phi(b) = y$. Therefore we have:

\begin{equation}
xy = \phi(a)\phi(b) = \phi(ab) = \phi(ba) = \phi(b)\phi(a) = yx
\end{equation}

And we have shown that $S_4$ is Abelian, which we know is not the case. Contradiction.
\end{proof}

%(b)
\subsection{}
I will show that $G$ contains an element of order 4.

\begin{proof}
We know that $S_4$ has an element of order 4. Call it $\alpha$. Because $\phi$ is onto, there exists $x \in G$ such that $\phi(x) = \alpha$. Moreover, by properties of isomorphisms, we have $|x|$ is divisible by 4, in particular $x^{4k} = e$ for some positive integer $k$. Using our theorem for order of powers of group elements, $x^k$ must necessarily have order 4.
\end{proof}

%(c)
\subsection{}

I will prove that $H$ is a subgroup of $G$.
\begin{proof}
First, notice that $e \in A_4$ and $\phi$ maps identity to identity. Therefore $e \in H$ (and $H$ is non-empty). Now, consider $a, b \in H$. I will show that $ab^{-1} \in H$ as well.

This is evident simply by properties of homomorphisms.

\begin{equation}
\phi(ab^{-1}) = \phi(a)\phi(b^{-1}) = \phi(a)[\phi(b)]^{-1}
\end{equation}

We already know $\phi(a) \in A_4$, and because $A_4$ is a group and $\phi(b) \in A_4$, it must be the case that $[\phi(b)]^{-1} \in A_4$ as well. And again because $A_4$ is a group, the whole thing must be in $A_4$. So by definition $ab^{-1} \in H$ and we are done.

\end{proof}

%(d)
\subsection{}

It is normal. Note that if $x \in G$ is part of $H$, then it is trivial. And if $x \notin H$, then clearly $x^{-1} \notin H$. It must be the case that both are mapped to something not in $A_4$, otherwise they would be in $A_4$. And we know that the product of two odd permutations and an even permutation is an even permutation. So we are done.

\end{document}